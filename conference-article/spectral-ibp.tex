\documentclass{article}
% Style file
\usepackage{assets/nips12submit_e,times}
% Basic packages
\usepackage[utf8]{inputenc}
\usepackage{amsmath, amsthm, amssymb, bm}
\usepackage{verbatim}
% More fun stuff
\usepackage{graphicx}%\graphicspath{{figures/}}
\usepackage{multicol}
\usepackage{multirow}
\usepackage{tabularx}
\usepackage[square,numbers]{natbib}
%\usepackage[usenames]{color}
\usepackage{mathrsfs}
\usepackage[colorlinks,citecolor=blue,urlcolor=blue,linkcolor=blue]{hyperref}
\usepackage{hypernat}
\usepackage{datetime}
\usepackage{textcomp}
\usepackage{assets/picins}
\usepackage{tikz}
\usetikzlibrary{shapes.geometric,arrows,chains,matrix,positioning,scopes,calc}
\tikzstyle{mybox} = [draw=white, rectangle]
\usepackage{booktabs}

% Algorithms, and some standard modifications
\usepackage{algorithm}
\usepackage{algorithmicx}
\usepackage{algpseudocode}
\renewcommand{\algorithmicrequire}{\textbf{Input:}}
\renewcommand{\algorithmicensure}{\textbf{Output:}}
\newcommand{\algrule}[1][.2pt]{\par\vskip.3\baselineskip\hrule height #1\par\vskip.3\baselineskip}

\title{
A spectral approximation to the Indian Buffet Process
}

\author{
James Robert Lloyd, Colorado Reed, Zoubin Ghahramani\\
Department of Engineering\\
Cambridge University\\
%\texttt{jrl44@cam.ac.uk, FIX ME}
%\And
%Coauthor \\
%Affiliation \\
%Address \\
%\texttt{email}
}

% The \author macro works with any number of authors. There are two commands
% used to separate the names and addresses of multiple authors: \And and \AND.
%
% Using \And between authors leaves it to \LaTeX{} to determine where to break
% the lines. Using \AND forces a linebreak at that point. So, if \LaTeX{}
% puts 3 of 4 authors names on the first line, and the last on the second
% line, try using \AND instead of \And before the third author name.

\newcommand{\fix}{\marginpar{FIX}}
\newcommand{\new}{\marginpar{NEW}}

\setlength{\marginparwidth}{1.25in}

%%%%%%%%%%%%%%%%%%%%%%%%%%%%%%%%%%%%%%%%%%%%%%%%%%%%%%%%%%
%%%% EDITING HELPER FUNCTIONS  %%%%%%%%%%%%%%%%%%%%%%%%%%%
%%%%%%%%%%%%%%%%%%%%%%%%%%%%%%%%%%%%%%%%%%%%%%%%%%%%%%%%%%

%% NA: needs attention (rough writing whose correctness needs to be verified)
%% TBD: instructions for how to fix a gap ("Describe the propagation by ...")
%% PROBLEM: bug or missing crucial bit 

%% use \fXXX versions of these macros to put additional explanation into a footnote.  
%% The idea is that we don't want to interrupt the flow of the paper or make it 
%% impossible to read because there are a bunch of comments.

%% NA's (and TBDs, those less crucially) should be written so 
%% that they flow with the text.

\definecolor{WowColor}{rgb}{.75,0,.75}
\definecolor{SubtleColor}{rgb}{0,0,.50}

% inline
\newcommand{\NA}[1]{\textcolor{SubtleColor}{ {\tiny \bf ($\star$)} #1}}
\newcommand{\LATER}[1]{\textcolor{SubtleColor}{ {\tiny \bf ($\dagger$)} #1}}
\newcommand{\TBD}[1]{\textcolor{SubtleColor}{ {\tiny \bf (!)} #1}}
\newcommand{\PROBLEM}[1]{\textcolor{WowColor}{ {\bf (!!)} {\bf #1}}}

% as margin notes

\newcounter{margincounter}
\newcommand{\displaycounter}{{\arabic{margincounter}}}
\newcommand{\incdisplaycounter}{{\stepcounter{margincounter}\arabic{margincounter}}}

\newcommand{\fTBD}[1]{\textcolor{SubtleColor}{$\,^{(\incdisplaycounter)}$}\marginpar{\tiny\textcolor{SubtleColor}{ {\tiny $(\displaycounter)$} #1}}}

\newcommand{\fPROBLEM}[1]{\textcolor{WowColor}{$\,^{((\incdisplaycounter))}$}\marginpar{\tiny\textcolor{WowColor}{ {\bf $\mathbf{((\displaycounter))}$} {\bf #1}}}}

\newcommand{\fLATER}[1]{\textcolor{SubtleColor}{$\,^{(\incdisplaycounter\dagger)}$}\marginpar{\tiny\textcolor{SubtleColor}{ {\tiny $(\displaycounter\dagger)$} #1}}}


%% For submission, make all render blank.
%\renewcommand{\LATER}[1]{}
%\renewcommand{\fLATER}[1]{}
%\renewcommand{\TBD}[1]{}
%\renewcommand{\fTBD}[1]{}
%\renewcommand{\PROBLEM}[1]{}
%\renewcommand{\fPROBLEM}[1]{}
%\renewcommand{\NA}[1]{#1}  %% Note, NA's pass through!

\theoremstyle{plain}

\def\ie{i.e.\ }
\def\eg{e.g.\ }
\def\indicator{\mathbb{I}}
\def\mean#1{\mathbb{E}[#1]}
\def\bigmean#1{\mathbb{E}\bigl[#1\bigr]}
\def\Bigmean#1{\mathbb{E}\Bigl[#1\Bigr]}
\def\cyl{\mathcal{Z}}
\def\eqae{=_{\mbox{\tiny a.e.}}}
\def\wrt{w.r.t.\ }
\def\ae{a.e.\ }
\def\equas{=_{\mbox{\tiny a.s.}}}
\def\equae{=_{\mbox{\tiny a.e.}}}
\def\iid{i.i.d.\ }
\def\Iid{I.i.d.\ }
%\def\inclusion{\jmath}
\def\inclusion{\mathcal{J}}
\def\inclusionX{\inclusion_{\xspace}}
\def\wstar{weak$^{\ast}$ }
% Symmetric difference
\def\symmdiff{\!\vartriangle\!}


% Indices

\def\indI{\mbox{\tiny I}}
\def\indJ{\mbox{\tiny J}}
\def\indK{\mbox{\tiny K}}
\def\indJI{\mbox{\tiny J$\setminus$I}}
\def\indE{\mbox{\tiny E}}
\def\indF{\mbox{\tiny F}}
\def\indD{\mbox{\tiny D}}
\def\indi{\mbox{\tiny{\{i\}}}}
\def\ind#1{\mbox{\tiny #1}}
\def\power{\mathcal{F}}
\def\powerD{\power(D)}
\def\powerE{\power(E)}
\def\powerL{\power(L)}
\def\parts{\mathcal{H}}
\def\partsQ{\parts(\mathcal{Q})}
\def\partsn{\parts[n]}
\def\partsN{\parts_{\infty}(\mathbb{N})}

% Spaces

\def\abstspace{\Omega}
\def\xspace{\mathcal{X}}
\def\yspace{\mathcal{Y}}
\def\tspace{\mathcal{T}}
\def\xspaceI{\xspace_{\indI}}
\def\xspaceJ{\xspace_{\indJ}}
\def\xspaceD{\xspace_{\indD}}
\def\xspaceE{\xspace_{\indE}}
\def\tspaceI{\tspace_{\indI}}
\def\tspaceJ{\tspace_{\indJ}}
\def\tspaceD{\tspace_{\indD}}
\def\tspaceE{\tspace_{\indE}}
\def\txspace{\tilde{\xspace}}
\def\yspaceI{\yspace_{\indI}}
\def\yspaceJ{\yspace_{\indJ}}
\def\yspaceD{\yspace_{\indD}}
\def\yspaceE{\yspace_{\indE}}
\def\txspace{\tilde{\xspace}}
\def\ttspace{\tilde{\tspace}}
\def\xI{x_{\indI}}
\def\xJ{x_{\indJ}}
\def\xD{x_{\indD}}
\def\xE{x_{\indE}}
\def\tImage{\Gamma}
\def\simp{\triangle}
\def\simpI{\simp_{\indI}}
\def\simpJ{\simp_{\indJ}}

\def\AI{A_{\indI}}
\def\AJ{A_{\indJ}}
\def\AD{A_{\indD}}
\def\AE{A_{\indE}}


%Space of Prob Measures
\def\pMeas{M}
%Space of Contents
\def\fMeas{N}
%Space of cont fcts
\def\cfspace{C}
%Hilbert space
\def\hilbert{\mathcal{L}^2}


\def\borelV{\borel_{V}}

% Set systems

\def\borel{\mathcal{B}}
\def\top{\mbox{Top}}

\def\borelI{\borel_{\indI}}
\def\borelJ{\borel_{\indJ}}
\def\borelD{\borel_{\indD}}
\def\borelE{\borel_{\indE}}
\def\tborel{\tilde{\borel}}
\def\abstfield{\mathcal{A}}
\def\field{\mathcal{C}}
\def\fieldI{\field_{\indI}}
\def\fieldJ{\field_{\indJ}}
\def\fieldK{\field_{\indK}}
\def\fieldD{\field_{\indD}}
\def\fieldE{\field_{\indE}}
\def\tfield{\tilde{\mathcal{C}}}
\def\Sfield{\mathcal{S}}
\def\SfieldI{\mathcal{S}_{\indI}}
\def\SfieldJ{\mathcal{S}_{\indJ}}
\def\SfieldD{\mathcal{S}_{\indD}}
\def\tSfield{\tilde{\mathcal{S}}}
\def\borelx{\borel_x}
\def\tborelx{\tborel_x}
\def\borelgamma{\tborel_{\tImage}}
%\def\borelth{\borel_{\theta}}
\def\borely{\borel_{y}}
%\def\borelT{\borel_t}
\def\borelT{\borel_{\tspace}}
\def\borelS{\borel_s}
\def\topI{\top_{\indI}}
\def\topJ{\top_{\indJ}}
\def\topD{\top_{\indD}}
\def\topE{\top_{\indE}}
\def\topV{\top_V}
\def\topws{\top_{\text{ws}}}
\def\topcc{\top_{\text{c}}}
\def\borelXI{\borel(\xspaceI)}
\def\borelXD{\borel(\xspaceD)}
\def\tborelX{\borel(\txspace)}
\def\borelTI{\borel(\tspaceI)}
\def\borelTD{\borel(\tspaceD)}
\def\tborelT{\borel(\ttspace)}


% Maps

\def\XI{X_{\indI}}
\def\Xi{X_{\ind{i}}}
\def\Xj{X_{\ind{j}}}
\def\ThetaI{\Theta_{\indI}}
\def\XJ{X_{\indJ}}
\def\ThetaJ{\Theta_{\indJ}}
\def\XD{X_{\indD}}
\def\ThetaD{\Theta_{\indD}}
\def\XE{X_{\indE}}
\def\ThetaE{\Theta_{\indE}}
\def\tX{\tilde{X}}
\def\tTheta{\tilde{\Theta}}

\def\SI{S_{\indI}}
\def\TI{T_{\indI}}

\def\rest{\phi}
\def\restD{\rest_{\indD}}
\def\restI{\rest_{\indI}}
\def\restJ{\rest_{\indJ}}
\def\restDI{\rest^{\indD}_{\indI}}
\def\inclusionD{\inclusion_{\indD}}
\def\inclusionE{\inclusion_{\indE}}
\def\projector{\mbox{pr}}
\def\projectorD{\projector_{\indD}}
\def\projectorI{\projector_{\indI}}
\def\projectorJI{\pi_{\indJ\indI}}
\def\indicator{\mathbb{I}}

% Projective systems

\def\po{\preceq}
\def\famD#1{{\lbrace #1 \rbrace}_{\indD}}
\def\famE#1{{\lbrace #1 \rbrace}_{\ind{I$\in$}\indE}}
\def\fJI{f_{\indJ\indI}}
\def\fKI{f_{\indK\indI}}
\def\fKJ{f_{\indK\indJ}}
\def\fII{f_{\indI\indI}}
\def\fI{f_{\indI}}
\def\fJ{f_{\indJ}}
\def\fK{f_{\indK}}
\def\fD{f_{\indD}}
\def\fDI{f^{\indD}_{\indI}}
\def\fDK{f^{\indD}_{\indK}}
\def\gJI{g_{\indJ\indI}}
\def\gI{g_{\indI}}
\def\gJ{g_{\indJ}}
\def\gD{g_{\indD}}
\def\hJI{h_{\indJ\indI}}
\def\hI{h_{\indI}}
\def\hJ{h_{\indJ}}
\def\hE{h_{\indE}}
\def\plim{\varprojlim}

% Measure and Conditionals

\def\abstmeasure{\mathbb{P}}
\def\P{P}
\def\PI{P_{\indI}}
\def\PJ{P_{\indJ}}
\def\PD{P_{\indD}}
\def\PE{P_{\indE}}
\def\PX{P_{\mbox{X}}}
\def\PTh{P_{\mbox{\Theta}}}
\def\PXI{P_{\XI}}
\def\PThI{P_{\mbox{\Theta}}}
\def\PXJ{P_{\mbox{X}}}
\def\PThJ{P_{\mbox{\Theta}}}
\def\PXD{P_{\mbox{X}}}
\def\PThD{P_{\mbox{\Theta}}}
\def\PXE{P_{\mbox{X}}}
\def\PThE{P_{\mbox{\Theta}}}
\def\tP{\tilde{P}}
\def\tPX{\tilde{P}_X}
\def\tPTh{\tilde{P}_{\Theta}}







\def\SI{S_{\indI}}
\def\SJ{S_{\indJ}}

\def\tk{\tilde{k}}
\def\kI{k_{\indI}}

\def\postkernel{k}
\def\indctr{\mathbbm{1}}
\def\sp#1{\left<#1\right>}


%Mallows
\def\Sr{\mathbb{S}_r}
\def\Sinf{\mathbb{S}_{\infty}}
\def\Sbar{\bar{\mathbb{S}}}
\def\DP#1{\mbox{DP}\left( #1 \right)}
\def\GP#1{\mbox{GP}\left( #1 \right)}
\def\x{\mathbf{x}}
\def\y{\mathbf{y}}



\def\tyspace{\tilde{\yspace}}
\def\tF{\tilde{F}}
\def\tT{\tilde{T}}
\def\tmodel{\tilde{\model}}
\def\tnu{\tilde{\nu}}


\def\PTheta{P^{\theta}}
\def\FTheta{F^{\theta}}
\def\TTheta{T^{\theta}}
\def\borelY{\borel_{\yspace}}

\def\PX{P^{x}}
\def\PXI{\PX_{\indI}}
\def\PXJ{\PX_{\indJ}}
\def\PXD{\PX_{\indD}}
\def\PThetaI{\PTheta_{\indI}}
\def\PThetaD{\PTheta_{\indD}}
\def\YI{Y_{\indI}}
\def\YJ{Y_{\indJ}}
\def\YD{Y_{\indD}}
\def\Tn{T^{(n)}}
\def\indexspace{\mathcal{W}}
\def\tyspace{\tilde{\yspace}}
\def\tY{\tilde{Y}}
\def\inclusionT{\inclusion_{\tspace}}
\def\tPTheta{\tilde{P}^{\theta}}
\def\tTn{\tilde{T}^{(n)}}
\def\inclusionY{\inclusion_{\yspace}}

\def\tyspace{\tilde{\yspace}}
\def\tF{\tilde{F}}
\def\tT{\tilde{T}}
\def\tmodel{\tilde{\model}}
\def\tnu{\tilde{\nu}}
\def\tOmega{\tilde{\abstspace}}
\def\tabstmeasure{\tilde{\abstmeasure}}
\def\model{\mathcal{P}}

\def\tf{\tilde{f}}
\def\tx{\tilde{x}}
\def\Dom{\mbox{Dom}}
\def\ty{\tilde{y}}

%Extra stuff - to be integrated in the future

\newcommand{\eqd}{\overset{\,_{\!d}}{=}}
\newcommand{\defn}[1]{\emph{#1}}

\newcommand{\Law}{\mathcal{L}}

\def\given{\,|\,}

\newcommand{\NonNegInts}{\mathbb{Z}_+}
\newcommand{\Nats}{\mathbb{N}}
\newcommand{\Rationals}{\mathbb{Q}}
\newcommand{\Reals}{\mathbb{R}}

\newcommand{\as}{\textrm{a.s.}}

\def\[#1\]{\begin{align}#1\end{align}}
\newcommand{\defas}{:=}

\newcommand{\Normal}{\mathcal{N}}
\newcommand{\dist}{\ \sim\ }

\newcommand{\kernel}{\kappa}
\newcommand{\kernelmatrix}{K}
\newcommand{\scalefactor}{s}
\newcommand{\lengthscale}{\ell}
\newcommand{\targets}{T}
\newcommand{\noise}{\sigma_\targets}
\newcommand{\pseudopoints}{\eta}
\newcommand{\inputpoints}{\xi}
\newcommand{\covhyppar}{\psi}
\newcommand{\logistic}{\phi}

\newcommand{\CompOrder}{\mathcal{O}}

\newtheorem{thm}{Theorem}%[section]
\newtheorem{lem}[thm]{Lemma}
\newtheorem{prop}[thm]{Proposition}
\newtheorem{cor}[thm]{Corollary}

\theoremstyle{definition}
\newtheorem{definition}[thm]{Definition}%[section]
\newtheorem{conj}{Conjecture}[section]
\newtheorem{exmp}{Example}[section]
\newtheorem{rem}[thm]{Remark}

\theoremstyle{remark}
%\newtheorem{rem}{Remark}
\newtheorem{note}{Note}
\newtheorem{case}{Case}

\def\Uniform{\mbox{\rm Uniform}}
\def\Bernoulli{\mbox{\rm Bernoulli}}
\def\ie{i.e.\ }
\def\eg{e.g.\ }
\def\iid{i.i.d.\ }
\def\simiid{\sim_{\mbox{\tiny iid}}}
\def\eqdist{\stackrel{\mbox{\tiny d}}{=}}
\def\GP{\mathcal{GP}}
\def\Transpose{^\textrm{T}}


\nipsfinalcopy % Uncomment for camera-ready version

\numberwithin{equation}{section}
\numberwithin{thm}{section}

% Document specific notation

\def\IBP{Z}
\def\Weights{A}
\def\Data{X}
\def\Graph{G}
\def\Adjacency{W}
\def\Degree{D}
\def\Laplacian{L}
\def\IBPPrior{\alpha}

\DeclareMathOperator*{\argmin}{\arg\!\min}

\begin{document}

\maketitle

\begin{abstract}
Can we use spectral methods to get a fast model based on the IBP?
Or maybe approximation schemes developed for spectral clustering can be carried over to the IBP case?
\end{abstract}

\section{An IBP model}

Consider a simple linear Gaussian model of the form
\begin{eqnarray}
\Weights & \dist & \Normal\,(0, \sigma_\Weights^2 I) \\
\IBP & \dist & \textrm{IBP}(\IBPPrior) \\
\Data & \dist & \Normal\,(\IBP \Weights, \sigma_\Data^2 I).
\end{eqnarray}
Copying from \cite{Griffiths2011} we should be guided by terms such as
\begin{eqnarray}
p\,(\Data | \IBP, \Weights) & \propto & \exp\,(-\textrm{tr}((\Data - \IBP\Weights)\Transpose(\Data - \IBP\Weights))) \\
p\,(\Data | \IBP) & \propto & |\IBP\Transpose\IBP + \frac{\sigma_\Data^2}{\sigma_\Weights^2}I|^{D/2} \exp\,(-\textrm{tr}(\Data\Transpose(I - \IBP(\IBP\Transpose\IBP + \frac{\sigma_\Data^2}{\sigma_\Weights^2}I)^{-1}\IBP\Transpose)\Data)).
\end{eqnarray}
For simplicity, we might initially consider simpler priors on $\IBP$ and or a maximum likelihood framework.

\section{Spectral clustering notes}

For a similarity matrix $\Graph$, with adjacency matrix $\Adjacency$ and diagonal degree matrix $\Degree$, the unnormalized graph Laplacian is defined as
\begin{equation}
\Laplacian = \Degree - \Adjacency.
\end{equation}
has the following property \citep{Luxburg2007}
\begin{eqnarray}
f' \Laplacian f = \frac{1}{2}\sum w_{ij}(f_i - f_j)^2
\end{eqnarray}
and the multiplicity of the zero eigenvalue of $\Laplacian$ is equal to the number of connected components in $\Graph$.

Define a vector $f$ as follows
\begin{equation}
f_i =
\begin{cases}
\sqrt{|\bar\IBP_1|/|\IBP_1|} & \textrm{if } i \in \IBP_1\\
-\sqrt{|\IBP_1|/|\bar\IBP_1|} & \textrm{if } i \in \bar\IBP_1\\
\end{cases}.
\end{equation}
Then we have
\begin{eqnarray}
f' \Laplacian f & = & \frac{1}{2}\Bigg(\sum_{i\in \IBP_1 j \in \bar\IBP_1} w_{ij} + \sum_{i\in \bar\IBP_1 j \in \IBP_1} w_{ij}\Bigg) \Bigg(\sqrt{\frac{|\bar\IBP_1|}{|\IBP_1}|} + \sqrt{\frac{|\IBP_1|}{|\bar\IBP_1|}}\Bigg)^2 \\
& = & \textrm{Cut}(\IBP_1, \bar\IBP_1)\Bigg(\frac{|\bar\IBP_1|}{|\IBP_1|} + \frac{|\IBP_1|}{|\bar\IBP_1|} + 2\Bigg) \\
& = & \textrm{Cut}(\IBP_1, \bar\IBP_1)\Bigg(\frac{|\bar\IBP_1| + |\IBP_1|}{|\IBP_1|} + \frac{|\IBP_1| + |\bar\IBP_1|}{|\bar\IBP_1|}\Bigg) \\
& = & \textrm{RatioCut}(\IBP_1, \bar\IBP_1) \times |\Graph|.
\end{eqnarray}

Additionally, $\sum f_i = 0$, and $||f||^2 = |\Graph|$.
To complete the link one then appeals to the Rayleigh--Ritz theorem to reveal that the solution to the relaxed problem is the second eigenvalue.

\section{Relations between the two}

Suppose we are trying to estimate $\Weights$ and $\IBP$ by maximum likelihood.
In particular, consider estimating the $k$th column of $\IBP$ and the corresponding row of $\Weights$ keeping all other parameters fixed.
Our objective can be stated as trying to minimise
\begin{equation}
||\Data - \IBP_{-k}\Weights_{-k} - \IBP_{k}\Weights_{k}||
\end{equation}
where $||.||$ is some distance metric on matrices (\ie the appropriate one to make this equivalent to maximum likelihood).

\fTBD{I have implicitly assumed that $z_{ij} \in \{-1,1\}$ so that we can model bias terms etc A minor point but might be useful for some forms of identifiability}
Let $\tilde X = \Data - \IBP_{-k}\Weights_{-k}$.
For a given $\IBP_k$, the maximum likelihood estimation of $\Weights$ is equivalent to minimising
\begin{equation}
\sum_{i \in \IBP_k}|\tilde x_i - \beta_k| + \sum_{i \in \bar\IBP_k}|\tilde x_i - \bar\beta_k|
\end{equation}
over $\beta_k$ and $\bar\beta_k$.
We can now create a link to spectral clustering.

Consider the following constant
\begin{eqnarray}
C & = & \sum_{i,j}|\tilde x_i - \tilde x_j| \\
  & = & \sum_{i,j \in \IBP_k}|\tilde x_i - \tilde x_j| + \sum_{i,j \in \bar\IBP_k}|\tilde x_i - \tilde x_j| + 2\sum_{i \in \IBP_k, j \in \bar\IBP_k}|\tilde x_i - \tilde x_j|
\end{eqnarray}
and then consider maximising $\sum_{i \in \IBP_1, j \in \bar\IBP_1}|\tilde x_i - \tilde x_j|$ over $\IBP_k$.
This can be recast as minimising quantities of the form $\sum_{i \in \IBP_k, j \in \bar\IBP_k}(1 - \alpha|\tilde x_i - \tilde x_j|)$ for any $\alpha > 0$.
For small enough $\alpha$ all summands will be positive and this can be phrased as a min cut problem \ie approximate spectral clustering.

Thus, spectral clustering is approximately equivalent to minimising $\sum_{i,j \in \IBP_k}|\tilde x_i - \tilde x_j| + \sum_{i,j \in \bar\IBP_k}|\tilde x_i - \tilde x_j|$.

Let $\hat\beta_k = \argmin_\beta \sum_{i \in \IBP_k}|\tilde x_i - \beta|$. Using this definition and the triangle inequality, we get the following
\begin{equation}
|\IBP_k|\sum_{i \in \IBP_1}|\tilde x_i - \hat\beta| \leq \sum_{i,j \in \IBP_1}|\tilde x_i - \tilde x_j| \leq 2|\IBP_k|\sum_{i \in \IBP_1}|\tilde x_i - \hat\beta_k|.
\end{equation}
\ie we can show that min cut is optimising a bound on the maximum likelihood objective.

The bound is only tight when the problem is degenerate \ie this is not yet a guarantee, just a heuristic.
The tightness of these bounds could be demonstrated in a probabilistic sense by assuming the data was generated by a linear binary model.

\section{A natural iterative algorithm}

In the maximum likelihood setting this is easy, the following algorithm can be justified using the arguments above.
\begin{itemize}
\item Find a binary clustering, $\IBP_1$, using spectral clustering applied to \Data
\item Fit maximum likelihood parameters to yield $\hat\Weights_1$
\item Obtain a new clustering, $\IBP_2$, by applying spectral clustering to $\Data - \IBP_1\Weights_1$. This is a sort of iterative conditional maximisation algorithm
\item Fit maximum likelihood parameters to yield $\hat\Weights_{1:2}$
\item Obtain a new clustering, $\IBP_{3}$, by applying spectral clustering to $\Data - \IBP_{1:2}\Weights_{1:2}$. This is a sort of iterative conditional maximisation algorithm
\item et cetera\ldots with looping, split merge equivalents and other nice things developed for samplers.
\end{itemize}

\section{Can we modify the spectral clustering algorithm}

Can we include Bayesian regularisation terms into the objective?
Can we make it more similar to maximum likelihood?
How can we introduce a prior on the partitions?

Many parts of the proof that spectral clustering does what it claims are highly intertwined.
On the surface, it looks like it will be difficult to 

\section{Literature review}

\TBD{Write notes here}

\cite{Griffiths2011}
\cite{Luxburg2007}
\cite{Cui2007}
\cite{Niu2010}
\cite{Niu2011}
\cite{Niu2012}

\section{$k$-means clustering as a generic preprocessing step}

I am currently using the approximate spectral clustering algorithm of \cite{Yan2009}.
Is this a good idea?
I will try to work out what is going on in this approximation.

\section{Next steps / questions}
\begin{itemize}
\item Can we modify the above argument to be more Bayesian?
\item In doing so, can we look into the min cut / spectral clustering approximation to find an appropriate graph Laplacian?
\item How different is this to orthogonal projections \citep{Cui2007}?
\item Try it? Choosing sensible parameters for the spectral clustering will not be entirely easy.
\item Can I make a guarantee about improving the marginal likelihood? Possibly not?
\end{itemize}

\section{Some relevant literature}

% Bibliography

%\newpage
\small{
\bibliographystyle{assets/natbib}
%\bibliographystyle{icml2013}
\bibliography{library}
}

\end{document}
